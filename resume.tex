% !TEX program = xelatex

\documentclass{resume}
%\usepackage{zh_CN-Adobefonts_external} % Simplified Chinese Support using external fonts (./fonts/zh_CN-Adobe/)
%\usepackage{zh_CN-Adobefonts_internal} % Simplified Chinese Support using system fonts

\begin{document}
\pagenumbering{gobble} % suppress displaying page number

\name{Runcheng Liu}

\basicInfo{
  (+86) 188-0112-9893 $\bullet$
  \href{mailto:runchengliu2000@gmail.com}{runchengliu2000@gmail.com} $\bullet$
  \href{https://github.com/lKingslayer}{GitHub} $\bullet$
  \href{https://www.facebook.com/profile.php?id=100012245712808}{facebook}
}

\section{Education}
\datedsubsection{\textbf{Tsinghua University}, Beijing, China}{09/2018 -- present}
B.S. in Mathematics and Physics


\section{Research Experience}
\datedsubsection{\href{http://nlp.csai.tsinghua.edu.cn/}{\textbf{Undergraduate Research Assistant in THUNLP Group}}}{09/2020 -- present}
Supervisor: Prof. \href{http://nlp.csai.tsinghua.edu.cn/~lzy/}{Zhiyuan Liu}, Tsinghua University
\\*\textbf{Project I: Network representation learning under multi-source heterogeneous data(Tencent Marketing Solution Project)}
\\*\textbf{Objective:} Combine the user data and multi-dimensional features, and use the maching learning algorithm to establish an end-to-end model which is able to recommend the ads to the right users automatically. 
\\*\textbf{Achievement:}
\begin{itemize}
  \item Designed and implemented a base model that combines a feature converter and the Heterogeneous Graph Transformer(HGT, the SOTA model on heterogeneous graph representation learning). 
  \item Pretrained the model on the training data, and fine-tuned the model on the downstream task(link prediction), obtained a better performance than the baseline(HGT).
\end{itemize}

\textbf{Project II: Pretrained language models(PLMs) for data-to-text generation}
\\*\textbf{Objective:} Investigate the impact of large PLMs in data-to-text generation.
\\*\textbf{Achievement:}
\begin{itemize}
  \item Demonstrated that PLMs such as BART achieved state-of-the-art result on DocRED dataset without explicitly encoding the graph structure.
  \item Applied PLMs on the Open Domain Event Text Generation(ODETG) task, and achieved better performance than the baseline. Demonstrated that PLMs is able to generate more informative text than the traditional encoder-retriever-decoder framework on this ODETG task.
\end{itemize}




\datedsubsection{\href{http://www.castu.tsinghua.edu.cn/publish/casen/index.html}{\textbf{Undergraduate Research Assistant in Institute for Advanced Study (IASTU)}}}{04/2019 -- 09/2019}
Supervisor: Prof. \href{http://astro.tsinghua.edu.cn/~xbai/}{Xuening Bai}, Tsinghua University
\\*\textbf{Project I: Numerical method for the electron magnetohydrodynamics(EMHD) equations(an Open Problem)}
\textbf{Objective:} Design an effective numerical approach for the EMHD equations.
\\*\textbf{Achievement:}
\begin{itemize}
    \item Designed an effective numerical algorithm under the Contrained Transport(CT) framework to capture the evolution of hall drift mode in the EMHD equations in 2 and 3 dimensions.
\end{itemize}


\section{Course Projects}
\datedsubsection{\textbf{Tsinghua} - Tsinghua, Beijing}{}
Artificial Neural Networks: \textbf{Exploration on the pre-trained representation of BERT and the selection of its significant subspace}
\begin{itemize}
    \item Conducted the experiment in \href{https://lena-voita.github.io/posts/mdl_probes.html}{Information-Theoretic Probing with MDL paper} by replacing ELMo with BERT, and justified the conclusions obtained in this paper successfully.
    \item Proposed an effective and simple method which selects the significant subspace of pre-trained representation based on gradients.
\end{itemize}

Big Data in Experimental Physics: \textbf{Classification of the types of OGLE Cephieds}
\begin{itemize}
    \item Obtained a simple and effective way using an external package to classify the types of OGLE Cephieds in a data-contest hosted on \href{https://data-contest.applysquare.com/challenges/pd2020}{crowdAI}.
\end{itemize}

Programming Fundamentals: \textbf{Simulation of temperature based on heat transfer}
\begin{itemize}
    \item Simulated the evolution of temperature based on heat transfer in three dimensioanl space with boundary conditions, and visualized the distribution of temperature using C language.
\end{itemize}


\section{Selected Honors \& Awards}
\begin{enumerate}
    \item Successful Participant, The Mathematical Contest in Modeling(MCM), 2020
    \item Comprehensive Excellence Award, Tsinghua University, 2019
    \item The First Prize, Chinese Physics Olympiad, 2017
    \item The First Prize, Chinese Physics Olympiad, 2016
\end{enumerate}


\section{Skills}
\begin{enumerate}
  \item \textbf{Programming Languages \& Tools:} \small Python, C/C++, Bash, Mathematica, MATLAB, R, Git, \LaTeX, PyTorch, TensorFlow
  \item \textbf{Language:} \small Chinese, Japanese, English 
\end{enumerate}


\end{document}
