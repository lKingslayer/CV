% !TEX program = xelatex

\documentclass{resume}
%\usepackage{zh_CN-Adobefonts_external} % Simplified Chinese Support using external fonts (./fonts/zh_CN-Adobe/)
%\usepackage{zh_CN-Adobefonts_internal} % Simplified Chinese Support using system fonts

\begin{document}
\pagenumbering{gobble} % suppress displaying page number

\name{Runcheng Liu}

\basicInfo{
  (+86) 188-0112-9893 $\bullet$
  \href{mailto:runchengliu2000@gmail.com}{runchengliu2000@gmail.com} $\bullet$
  \href{https://github.com/lKingslayer}{GitHub} $\bullet$
  \href{https://www.facebook.com/profile.php?id=100012245712808}{facebook}
}

\section{Education}
\datedsubsection{\textbf{Tsinghua University}, Beijing, China}{09/2018 -- present}
B.S. in Mathematics and Physics

\section{Research Experience}
\datedsubsection{\href{http://www.castu.tsinghua.edu.cn/publish/casen/index.html}{\textbf{Research Intern in Institute for Advanced Study (IASTU)}}}{04/2019 -- 09/2019}
Supervisor: Prof. \href{http://astro.tsinghua.edu.cn/~xbai/}{Xuening Bai}, Tsinghua University
\\*Project I: Numerical method for the electron magnetohydrodynamics(EMHD) equations.(an Open Problem)
\begin{itemize}
    \item Designed an effective numerical algorithm within the Contrained Transport(CT) framework to capture the evolution of hall drift mode in the EMHD equations in 2 and 3 dimensions.
\end{itemize}

\datedsubsection{\href{http://nlp.csai.tsinghua.edu.cn/}{\textbf{Research Assistant in THUNLP Group}}}{09/2020 -- present}
Supervisor: Prof. \href{http://nlp.csai.tsinghua.edu.cn/~lzy/}{Zhiyuan Liu}, Tsinghua University
\\*Project I: Recommendation system of tencent

\section{Course Projects}
\datedsubsection{\textbf{Tsinghua} - Tsinghua, Beijing}
\begin{itemize}
    \item Programming Fundamentals: Simulated the evolution of temperature based on heat transfer in three dimensioanl space with boundary conditions. Visualized the distribution of temperature using C language.
    \item Big Data in Experimental Physics:  Obtained a simple and effective way using an external package to classify the types of OGLE Cephieds in a data-contest hosted on \href{https://data-contest.applysquare.com/challenges/pd2020}{crowdAI}.
    \item Artificial Neural Networks: Conducted the experiment in \href{https://lena-voita.github.io/posts/mdl_probes.html}{Information-Theoretic Probing with MDL paper} by replacing ELMo with BERT, and justified the conclusions obtained in this paper successfully. Proposed an effective and simple method which selects the significant subspace of pre-trained representation based on gradients.
\end{itemize}



\section{Selected Honors \& Awards}
\begin{enumerate}
    \item Successful Participant, The Mathematical Contest in Modeling(MCM), 2020
    \item Comprehensive Excellence Award, Tsinghua University, 2019
    \item The First Prize, Chinese Physics Olympiad, 2017
    \item The First Prize, Chinese Physics Olympiad, 2016
\end{enumerate}

\section{Skills}
\textbf{Programming Languages:} \small Python, C/C++, Bash, MATLAB, R

\textbf{Tools and Frameworks:} \small Git, \LaTeX, PyTorch, TensorFlow


\end{document}
